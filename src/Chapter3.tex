\section{Manually editing tracks in Mastodon. TrackScheme.}

One of the key feature of Mastodon, the implementation of which made our lives extremely hard and rich, is the ability to edit manually at any time any spot or link within a data model that can contains billions of them, while retaining a good and pleasant response time from the software.
I am not even speaking of the undo/redo mechanism, which getting right was also very interesting. 

In this short tutorial, we will show how to do just that, from existing tracking data, the one we generated in the previous section. 
We will use it as an opportunity to present TrackScheme, the track visualizer of Mastodon, as well as introduce several useful editing features such as undo/redo mechanism mentioned above.

Manual editing is often \textbf{not desirable}, first because it is not objective which might be detrimental in situations where the performance of an algorithm is evaluated, or when an unknown motion characteristics is investigated. 
Second, because it does not look like a realistic solution when \textit{e.g.} thousands of cells are to be followed over thousands of time-points.
Regarding the latter however, several courageous individuals sacrificed their time, energy and sometimes sanity in doing so, for the sake of finally getting a scientific answer when there was no other working solutions. 
See for instance~\cite{MaMuT} and~\cite{McDole2018}.
If you are going this way, Mastodon aims at providing tools to do so, in the fastest and least painful way possible. 
Also, you can combine the fully automated approach of the previous chapter with manual editing to correct hopefully a few numbers of mistakes.
Finally, there is also a semi-automated tracking tool, that we will introduce later.


\subsection{TrackScheme, the lineage view and editor.}

First, load or retrieve the tracking data of the previous chapter. 
You should have a few hundreds of tracks. 
To open a TrackScheme view window, press the \menu{trackscheme} button on the main window (Figure~\ref{fig:MastodonMainWindow} page~\pageref{fig:MastodonMainWindow}).
This window should show up. 

The TrackScheme view is a special way of displaying tracks. 
If you are familiar with TrackMate, you will see that we brought the same kind of features here, but scaled to large data.
You can think of TrackScheme as a workbench for tracks, where you will edit, cut, stitch and rename them.
It displays a kind of "track map", where a track is laid on a panel, arranged vertically over time, as a Parisian subway train map. 
Tracks are displayed hierarchically, discarding the spatial location of each spot. 
Each track is laid out going through time from top to bottom.
One horizontal line corresponds to a single time-point in the movie.
One vertical column corresponds to a single track, that is all the spots that are connected by links over time, including divisions and merges.
It is a great tool particularly to study and edit cell lineages.

When opened, the view in TrackScheme is scaled so that the full data is shown, all tracks over all time-points. 
Even on our small dataset, most of the tracks and spots appear as single lines or dense boxes.
To see the details of each track, you need to zoom in and navigate around the data.
The navigation actions their mappings are listed in the Table~\ref{tab:MastodonTrackSchemeNavigationKeys} below.

\begin{table}[!htbp]
    \centering
    \caption{Default navigation key-bindings for Mastodon-TrackScheme views.}
    \begin{tabulary}{\textwidth}{L|J}
    
    \toprule
    \textbf{Action}                 & \textbf{Key}              
    \\ \midrule
    
    \multicolumn{2}{c}{\textit{View.}}
    \\ \midrule
    
    Move around.                    & \keys{Right-click} and \keys{Drag} or \keys{Mouse-wheel}
    \\ \midrule
    
    Zoom / unzoom in X.             & \keys{\shift+Mouse-wheel}
    \\ \midrule
    
    Zoom / unzoom in Y.             & \keys{\ctrl+Mouse-wheel}
    \\ \midrule
    
    Zoom / unzoom in X \& Y.         & \keys{\ctrl+\shift+Mouse-wheel}
    \\ \midrule

    Full zoom, full unzoom.         & Press \keys{Z}. The view zoom at max level to the mouse location. Pressing  \keys{Z} again to unzoom fully.
    \\ \midrule
    
    Zoom in a box.                  & Press and hold \keys{Z} then drag a box. The view will zoom to the box.                            
    \\ \bottomrule

\end{tabulary}

    \label{tab:MastodonTrackSchemeNavigationKeys}
\end{table}

Try to zoom in until you can see the label of a few spots.
TrackScheme implements adaptive level of details depending on the zoom level, so that we can accommodate plotting a large amount of data without compromising the responsiveness of Mastodon too much.
On the finer level of details, spots are plotted as circles, with the spot label shown.
As you zoom out, they becomes just empty circles, then points, then they disappear to only show the track as a line.
When the zoom level is so low that several tracks coalesce, they are drawn as a gray box. 


\subsection{The focus and the spot labels.}

You probably noticed that some spots are painted differently from their neighbors. 
Mastodon manages three special collections of spots and links to facilitate making sense of the data across views:
\begin{myitemize}
    \item the \textbf{selection}, painted in green, that manages a classical selection of spots and links;
    \item the \textbf{highlight}, used to highlight the spot or link currently under the mouse;
    \item the \textbf{focus}, particularly used in TrackScheme, to indicate what spot is currently focused on by the keyboard interaction.
\end{myitemize}

We will first present the focus (see Table~\ref{tab:MastodonTrackSchemeFocusKeys}). 

\begin{table}[htbp]
    \centering
    \caption{Default key-bindings for the focus in TrackScheme and BDV views.}
    \begin{tabulary}{\textwidth}{L|J}
    
    \toprule
    \textbf{Action}                 & \textbf{Key}              
    \\ \midrule
    
    \multicolumn{2}{c}{\textit{Navigation with the Focus.}}
    \\ \midrule
    
    Follow a spot across time within a track with the focus.      & \keys{\arrowkeyup} and \keys{\arrowkeydown}
    \\ \midrule

    Jump to the beginning of a branch.                              & \keys{\Alt + \arrowkeyup}
    \\ \midrule
    
    Jump to the end of a branch.                                    & \keys{\Alt + \arrowkeydown}
    \\ \midrule

    Jump to the beginning of another branch.                              & \keys{\ctrl + \Alt + \arrowkeyup}
    \\ \midrule

    Jump to the end of another branch.                                    & \keys{\ctrl + \Alt + \arrowkeydown}
    \\ \midrule
    
    \multicolumn{2}{c}{\textit{Focus in TrackScheme.}}
    \\ \midrule
    
    Move the focus around from one track or one track branch to another.  & \keys{\arrowkeyleft} and \keys{\arrowkeyright}
    \\ \midrule
    
    Edit the label of the focused spot.             & \keys{\return}
    \\ \bottomrule
  
\end{tabulary}

    \label{tab:MastodonTrackSchemeFocusKeys}
\end{table}


\subsubsection{Moving the focus.}

The spot that has the focus is painted with a thick and dashed contour. 
Only one spot can have the focus and it is set by mouse or keyboard interaction.
Click on one spot in TrackScheme when fully zoomed, and move across track and time with the arrow cursor keys \keys{\arrowkeyleft} \keys{\arrowkeyright} \keys{\arrowkeyup} and \keys{\arrowkeydown}.
If the focus reaches the border of the window, the view will be moved to follow it.
You can think of the focus as the caret in a text editor. 
It is meant to facilitate keyboard interaction.

\subsubsection{Editing the spot labels.}

By the way, the focus is used to rename individual spots.
Zoom in so that we can see the label of the spots and move the focus to a spot. 
Press \keys{\return}.
A small editing box appears inside the spot and lets you change its label.

By default the spot label display the spot ID. 
If you edit the label then it shows the new label you entered.
Editing the label of a spot does not affect its ID or any other properties. 
The spot label is just a convenient text field that you can use to annotate cells and search them later.
Only the spots have a label. 
The links don't.

\subsubsection{The order of tracks in TrackScheme.}

The spot labels also control how the tracks are ordered in TrackScheme.
We said before that in TrackScheme all the spatial information is discarded. 
However the tracks are laid out in a deterministic order.
This order is set by the label of of a track. 

There is no special structure to follow individual tracks in Mastodon. 
For this, we simply use the first spot of a track. 
So when we speak of the label of a track, we simply means the label of the first spot (in time) in the track. 

The tracks are arranged from left to right following the alphanumerical order of the track labels.
So you can change the tracks arrangement by editing theirs first spot's label.
For instance a track named \texttt{A} will be laid out to the left of a tracks named \texttt{B}, and a track named \texttt{D9} will be put to the left of a track named \texttt{D10}.

If you change the track labels now, the tracks will not move immediately in TrackScheme.
For the new arrangement to happen, you need to either open another TrackScheme window, or to edit the data, which we will see soon.

\subsubsection{The focus in BDV views.}

The focus also works in the BDV views.
In these views, the spot that has the focus is also painted with a thick dashed circle.
Parenthetically, notice that the focus is shared across all opened views. 
If you have a TrackScheme view and a BDV view opened, and that they show show roughly the same spot, setting the focus in one view will update the spot display in the other views.

When you are in a BDV view and have set the focus, you can also navigate with the arrow keys.
But in these views, only navigating in time (\keys{\arrowkeyup} and \keys{\arrowkeydown}) will work.
Also you cannot edit the spot labels in a BDV view.



\subsection{Linking several views together.}

the locks.

navigating over time within a track.



\subsection{Deleting individual spots and links.}

\subsection{The selection.}

adding removing to the selection

editing with the selection (deleting)

\subsection{Manually adding spots and linking them.}

\subsection{The undo/redo mechanism.}




