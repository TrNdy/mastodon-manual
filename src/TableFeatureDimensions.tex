\begin{tabular}{ l | l | p{0.125\textwidth} | p{0.50\textwidth} }

    \toprule
    \textbf{Dimension} &
    \textbf{Name} &
    \textbf{Example units (when spatial units are µm)} &
    \textbf{Description}
    
    \\ \midrule
    
    \texttt{NONE} &
    None &
    ø &
    Used for dimensionless quantities, such as frame position, number of things, \etc 
    \\ \midrule
    
    \texttt{LENGTH} & 
    Length &
    µm &
    For quantities about the length of objects. 
    For instance radius or distance between objects.
    \\ \midrule
    
    \texttt{POSITION} & 
    Position &
    µm &
    Dimension for feature that report a position.
    Different from \texttt{LENGTH} so that for objects with small lengths at large positions, quantities are plotted separately.
    \\ \midrule
    
    \texttt{TIME} &
    Time &
    frame &
    For quantities that report a delay, a duration or the timing of an event.
    Because Mastodon does not deal with physical units for time, quantities formed with the time dimension always use the \texttt{frame} unit.
    \\ \midrule
    
    \texttt{VELOCITY} & 
    Velocity & 
    µm / frame &
    For quantities that report a speed or a velocity.
    \\ \midrule
    
    \texttt{RATE} &
    Rate &
    / frame &
    For quantities that report a change per units of time.
    \\ \midrule
    
    \texttt{ANGLE} &
    Angle &
    Radians &
    Measures of angles. 
    In Mastodon, all angles are in \textbf{radians}.
    \\ \midrule
    
    \texttt{STRING} &
    NA &
    ø &
    For non-numeric features.
    \\ \midrule
    
    \texttt{INTENSITY} &
    Intensity &
    Counts &
    For quantities based on pixel values.
    For instance the mean intensity within a spot.
    \\ \midrule
    
    \texttt{INTENSITY\_SQUARED} &
    Intensity² &
    Counts² &
    For quantities based on pixel intensity squared.
    For instance the variance of the mean within a spot.
    \\ \midrule
    
    \texttt{QUALITY} &
    Quality &
    ø &
    This dimension is used by spot detectors.
    There is a special feature called \textbf{Detection quality}, that stores for each spot they detect automatically a measure of quality or confidence in their detection.
    See section~\ref{Detection_Cells_DoG_Detector}.
    \\ \midrule
    
    \texttt{COST} &
    Cost &
    ø &
    This dimension is used by spot linking algorithms.
    There is a special feature called \textbf{Link cost} used in the estimation phase.
    It stores for each link the cost that the linker computes for it in the estimation phase.
    These costs are then used in the association phase to retrieve the best set of links.
    \\ \bottomrule

\end{tabular}