\section{Numerical features and tags. The table view.}

Mastodon is a tracking and lineaging tool.
Its output is a collection of tracks, and the analysis of these tracks to yield statistics on \eg velocity, displacement \etc is carried out in another software package such as MATLAB or Python.
Nonetheless you will find in Mastodon tools to compute \textit{numerical features} on data item. 
Numerical features are numbers that can be calculated on spots, links and tracks of the data. 
For instance there are feature for the number of links that touch a spot, or the displacement of a link or the number of spots in a track.
You can find them within Mastodon because it is convenient, but also because they are very useful for the interactive exploration of your data. 
Coupled with feature-based coloring, the display and sorting of values in the table view and the selection creator tool, they can considerably accelerate and facilitate making sense of the data.

Numerical features are numbers that classically relate to a physical quantity.
When we need to \textit{categorize} items, we rely on \textit{tags}.
We describe them just below. 
This chapter will also show you how to compute numerical features and create a coloring view from the feature values and  tags.
Doing so, we will introduce the third kind of data view in Mastodon: the data tables.


\subsection{Tags and tag-sets.}

As we said above, every-time you need to categorize certain data items, or need to visualize categories, you should rely on tags.
Let's suppose that you are investigating the trajectories of cells in a developing embryo from an early stage to a stage where the embryo is polarized.
Some cells will migrate to the anterior part, some others to the posterior part, \etc. 
You might want to tag cell tracks with the \texttt{Anterior} or \texttt{Posterior} tag, to investigate where does these cell come from in the early embryo.
Or let's say that you are curating the results of the automated tracking on a large images. 
The tracking results might have some inaccuracies, and you want to correct them for important tracks.
Because there is a lot of tracks, you share the workload with some colleagues. 
You work asynchronously with them, editing the Mastodon file one after another. 
Doing so, you can use tags in Mastodon to mark some tracks as reviewed by you.
Your colleagues will use a tag for themselves, to ensure that no two scientists are reviewing the same track twice.
All the cells that are not tagged in this categorisation are still waiting to be reviewed. 

In Mastodon, a categorization corresponds to a \textbf{tag-set}.
A tag-set defines a property that can have a discrete number of possible values, or \textbf{tags}.
In the first of the two examples above, \texttt{Location} would be a tag-set to specify the location of cells.
\texttt{Anterior} and \texttt{Posterior} would be two tags belonging to the \texttt{Location} tag-set.
In the second example, \texttt{Reviewed by} would be a tag-set, and \texttt{Mette}, \texttt{Pavel}, \texttt{Tobias} and \texttt{Jean-Yves} would be 4 tags of this tag-set.

You can assign tags to spots and links.
To assign a tag to a whole track, you have to assign this tag to all the spots and links of this track.
One data item (a spot or a ling) can have 1 or 0 tags per existing tag-set.
But they can be categorized by as many tag-sets as there is.
For instance, a spot can have the tag \texttt{Anterior} in the \texttt{Location} tag-set, and the tag \texttt{Pavel} in the \texttt{Reviewed by} tag-set. 
Or it can be not tagged in the \texttt{Reviewed by} tag-set.
But it cannot have both the tag \texttt{Mette} and the tag \texttt{Tobias} because they belong to the same tag-set.
Each tag-set works independently, and clearing a tag-set does not affect the others even for one data item.
With this, let's see how to create tag-sets.


\subsubsection{Creating tag-sets.}





\subsubsection{Assigning tags to data items.}

\subsubsection{Coloring views by tag-sets.}


\subsection{Numerical features.}

\subsubsection{Feature computation.}

\subsubsection{Coloring views by numerical features.}

\subsubsection{Accelerating feature computation with the update mechanism.}


\subsection{Tags, numerical features and saving the data to disk.}

\subsection{The data table views.}

