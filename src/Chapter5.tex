\section{Numerical features and tags. The table view.}

Mastodon is a tracking and lineaging tool.
Its output is a collection of tracks, and the analysis of these tracks to yield statistics on \eg velocity, displacement \etc is carried out in another software package such as MATLAB or Python.
Nonetheless you will find in Mastodon tools to compute \textit{numerical features} on data item. 
Numerical features are numbers that can be calculated on spots, links and tracks of the data. 
For instance there are feature for the number of links that touch a spot, or the displacement of a link or the number of spots in a track.
You can find them within Mastodon because it is convenient, but also because they are very useful for the interactive exploration of your data. 
Coupled with feature-based coloring, the display and sorting of values in the table view and the selection creator tool, they can considerably accelerate and facilitate making sense of the data.

Numerical features are numbers that classically relate to a physical quantity.
When we need to \textit{categorize} items, we rely on \textit{tags}.
