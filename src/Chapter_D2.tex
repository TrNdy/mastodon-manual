\section{The graph data structure of Mastodon.}

The \emph{mastodon-graph} Java package can be used to implement directed graphs with small memory footprint.
Vertices and edges are not stored as individual objects.
Instead vertex and edge data is laid out in a primitive \texttt{byte[]} array and accessed via proxy objects.
This chapter describes some internals of the \emph{trackmate-graph} package.

\pgfdeclarelayer{background}
\pgfsetlayers{background,main}

\tikzset{every picture/.append style={>=latex'}}

\tikzstyle{obj layout matrix}=[
  	matrix anchor=north west,
		inner sep=0pt,
		outer ysep=0pt,
		nodes={
			inner sep=3.5pt,
			draw,
			minimum height=7mm,
			text width=30mm,
			font=\ttfamily
		}
	]
	
\tikzstyle{obj layout indices}=[
  	matrix anchor=north east,
		xshift=4mm,
		nodes={
			%draw=gray!20,
			minimum height=7mm,
			text width=6mm,
			align=right,
			font=\ttfamily
		},
		text=blue
	]
	
\tikzstyle{edge layout indices}=[
		obj layout indices,
		text=red
	]
	
\tikzstyle{obj layout byte indices}=[
		obj layout indices,
		xshift=-2.5mm,
		yshift=2mm,
		text=gray
	]


% -------------------------------------
\subsection{Memory layout.}
% -------------------------------------

Each vertex and each edge maps to a contiguous portion of a \texttt{byte[]} array.
The size of a portion (\emph{elements}) is fixed for a particular \texttt{AbstractVertex} or \texttt{AbstractEdge} subclass.
Each \emph{element} starts with a fixed part that represents the graph structure and then additional payload used by the subclass to describe some vertex attributes, etc.
 
There is one \texttt{byte[]} array that stores all vertices, and one \texttt{byte[]} array that stores all edges.
References between these arrays are in the form of \emph{element indices}.
Whether these are indices refer to elements in the vertex or in the edge memory array is clear from the context.

% -------------------------------------
\subsubsection{Vertex layout.}
% ------------------------------------
The following diagram illustrates the layout of vertices in a \texttt{byte[]} array:
\begin{center}
	\begin{tikzpicture}[
			scale=0.8, every node/.style={scale=0.8},
			inner frame xsep=0mm,
			inner frame ysep=0mm,
			background rectangle/.style={
	    	fill=black!10
	    },
	    %show background rectangle
		]
		\newcommand{\spot}[4]{
		  \matrix (l) at (l.south west) [obj layout matrix]
		  {
		    \node (sl1) {FIRST\_IN\_EDGE}; \\
		    \node (sl2) {FIRST\_OUT\_EDGE}; \\
		    \node (sl3) [minimum height=20mm] {}; \\
		  };
		  
		  \draw
		  	[decoration={brace,amplitude=0.5em},decorate,thick]
		 			($(sl1.north east) + (3mm, -0.2mm)$) -- 
					($(sl2.south east) + (3mm, 0.2mm)$)
					node [anchor=west, midway, yshift=-1pt, xshift=2mm, font=\ttfamily] {AbstractVertex};
					
		  \draw
		  	[decoration={brace,amplitude=0.5em},decorate,thick]
		 			($(sl1.north east) + (1mm, -0.2mm)$) -- 
					($(sl3.south east) + (1mm, 0.2mm)$)
					node [anchor=west, midway, yshift=-1pt, xshift=2mm, font=\ttfamily] {Spot};
					
			\matrix (b) [obj layout byte indices]
			at (l.north west)
			{
		    \node {#2}; \\
		    \node {#3}; \\
		    \node {#4}; \\
			};
			\matrix (i) [obj layout indices]
			at (b.north west)
			{
		    \node {#1}; \\
			};
		}
		\coordinate (l);
		\spot{0}{0}{4}{8}
		\spot{1}{57}{61}{65}
	  \matrix (l) at (l.south west) [obj layout matrix]
	  {
	    \node (sl1) {FIRST\_IN\_EDGE}; \\
	    \node (sl2) [draw=none] {...}; \\
	  };
	  \draw (sl2.north west) -- (sl2.south west);
	  \draw (sl2.north east) -- (sl2.south east);
	  \draw (sl2.north west) -- (sl2.north east);
		\matrix (b) [obj layout byte indices]
		at (l.north west)
		{
	    \node {114}; \\
		};
		\matrix (i) [obj layout indices]
		at (b.north west)
		{
	    \node {2}; \\
		};
	\end{tikzpicture}
\end{center}
\noindent
In the left-most column, \emph{element indices} are shown (in blue), followed by byte indices (in grey).
In the example, each element of the \texttt{AbstractVertex} subclass \texttt{Spot} requires $57$ bytes to store.
The data for the $i$th \texttt{Spot} starts at byte $i \cdot 57$.
The fixed \texttt{AbstractVertex} part comprises element indices \texttt{FIRST\_IN\_EDGE} and \texttt{FIRST\_OUT\_EDGE}, occupying 4 bytes each.
The remaining 49 bytes are \texttt{Spot} attributes.

\texttt{FIRST\_IN\_EDGE} is the element index (in the edge memory array) of the first \emph{incoming} edge, i.e., an edge pointing to this vertex. The remaining incoming edges of the same vertex are stored as a linked list in the edge memory as described below.
If this vertex does not have any incoming edges \texttt{FIRST\_IN\_EDGE} is $-1$.

Similarly, \texttt{FIRST\_OUT\_EDGE} is the element index of the first \emph{outgoing} edge, i.e., an edge starting from this vertex. The remaining outgoing edges of the same vertex are stored as a linked list in the edge memory as described below.
If this vertex does not have any outgoing edges \texttt{FIRST\_OUT\_EDGE} is $-1$.

% -------------------------------------
\subsubsection{Edge layout.}
% -------------------------------------
The following diagram illustrates the layout of edges in a \texttt{byte[]} array:
\begin{center}
	\begin{tikzpicture}[
			scale=0.8, every node/.style={scale=0.8},
			inner frame xsep=0mm,
			inner frame ysep=0mm,
			background rectangle/.style={
	    	fill=black!10
	    },
	    %show background rectangle
		]
		\newcommand{\edge}[6]{
		  \matrix (l) at (l.south west) [obj layout matrix]
		  {
		    \node (sl1) {SOURCE}; \\
		    \node (sl2) {TARGET}; \\
		    \node (sl3) {NEXT\_SOURCE\_EDGE}; \\
		    \node (sl4) {NEXT\_TARGET\_EDGE}; \\
		    \node (sl5) [minimum height=12mm] {}; \\
		  };
		  
		  \draw
		  	[decoration={brace,amplitude=0.5em},decorate,thick]
		 			($(sl1.north east) + (3.5mm, -0.2mm)$) -- 
					($(sl4.south east) + (3.5mm, 0.2mm)$)
					node [anchor=west, midway, yshift=-1pt, xshift=2mm, font=\ttfamily] {AbstractEdge};
					
		  \draw
		  	[decoration={brace,amplitude=0.5em},decorate,thick]
		 			($(sl1.north east) + (1mm, -0.2mm)$) -- 
					($(sl5.south east) + (1mm, 0.2mm)$)
					node [anchor=west, midway, yshift=-1pt, xshift=3mm, font=\ttfamily] {Edge};
					
			\matrix (b) [obj layout byte indices]
			at (l.north west)
			{
		    \node {#2}; \\
		    \node {#3}; \\
		    \node {#4}; \\
		    \node {#5}; \\
		    \node {#6}; \\
			};
			\matrix (i) [edge layout indices]
			at (b.north west)
			{
		    \node {#1}; \\
			};
		}
		\coordinate (l);
		\edge{0} {0} {4} {8}{12}{16}
		\edge{1}{24}{28}{32}{36}{40}
	  \matrix (l) at (l.south west) [obj layout matrix]
	  {
	    \node (sl1) {SOURCE}; \\
	    \node (sl2) [draw=none] {...}; \\
	  };
	  \draw (sl2.north west) -- (sl2.south west);
	  \draw (sl2.north east) -- (sl2.south east);
	  \draw (sl2.north west) -- (sl2.north east);
		\matrix (b) [obj layout byte indices]
		at (l.north west)
		{
	    \node {48}; \\
		};
		\matrix (i) [edge layout indices]
		at (b.north west)
		{
	    \node {2}; \\
		};
	\end{tikzpicture}
\end{center}
\noindent
In the left-most column, \emph{element indices} are shown (in red), followed by byte indices (in grey).
In the example, each element of the \texttt{AbstractEdge} subclass \texttt{Edge} requires $24$ bytes to store.
The data for the $i$th \texttt{Edge} starts at byte $i \cdot 24$.
The fixed \texttt{AbstractEdge} part comprises element indices
	\texttt{SOURCE},
	\texttt{TARGET},
	\texttt{NEXT\_SOURCE\_EDGE}, and
	\texttt{NEXT\_TARGET\_EDGE}, occupying 4 bytes each.
The remaining 8 bytes are \texttt{Edge} attributes.

\begin{itemize}
    \item \texttt{SOURCE} is the element index (in the vertex memory array) of the vertex from which this edge starts.

    \item \texttt{TARGET} is the element index (in the vertex memory array) of the vertex to which this edge points.

    \item \texttt{NEXT\_SOURCE\_EDGE} is the element index (in the edge memory array) of the next outgoing edge of the source vertex, \ie, the next edge that has the same \texttt{SOURCE}. 
    If there is no such edge then \texttt{NEXT\_SOURCE\_EDGE} is $-1$.

    \item  \texttt{NEXT\_TARGET\_EDGE} is the element index (in the edge memory array) of the next incoming edge of the target vertex, \ie, the next edge that has the same \texttt{TARGET}.
    If there is no such edge then \texttt{NEXT\_TARGET\_EDGE} is $-1$.

\end{itemize}




% -------------------------------------
\subsubsection{Example}
% -------------------------------------

\tikzstyle{spot matrix}=[
  	matrix anchor=north west,
		inner sep=0pt,
		outer ysep=0pt,
		nodes={
			inner sep=3.5pt,
			draw,
			minimum height=5mm,
			text width=9mm,
			align=right,
		}
	]

\tikzstyle{incoming edge}=[
		green!50!black,
		thick,
		->,
		shorten >=1mm,
		shorten <=1mm,
		text=green!50!black,
		anchor=east
	]

\tikzstyle{outgoing edge}=[
		orange!70!black,
		thick,
		->,
		shorten >=1mm,
		shorten <=1mm,
		text=orange!50!black,
		anchor=east
	]

\tikzstyle{incoming bg}=[
		fill=green!10,
		inner sep=0pt
	]

\tikzstyle{outgoing bg}=[
		fill=orange!10,
		inner sep=0pt
	]

\newcommand{\spot}[4]{
  \matrix (s) at (s.south west) [spot matrix]
  {
    \node (s#4c1) [text=red] {#1}; \\
    \node (s#4c2) [text=red] {#2}; \\
    \node (s#4c3) [minimum height=13.5mm] {}; \\
  };
	\node at (s#4c1.west) [text width=2mm, draw=none, xshift=2mm, text=black] {$#4$};
	\node at (s#4c1.west) [text width=3mm, draw=none, xshift=-1mm, text=blue, font=\scriptsize] {#3};
}

\newcommand{\edge}[6]{
  \matrix (e) at (e.south west) [spot matrix]
  {
  	\node (e#6c1) [text=blue] {#1}; \\
    \node (e#6c2) [text=blue] {#2}; \\
    \node (e#6c3) [text=red]  {#3}; \\
    \node (e#6c4) [text=red]  {#4}; \\
    \node (e#6c5) [minimum height=9mm] {}; \\
  };
	\node at (e#6c1.west) [text width=2mm, draw=none, xshift=2mm, text=black] {$#6$};
	\node at (e#6c1.west) [text width=3mm, draw=none, xshift=-1mm, text=red, font=\scriptsize] {#5};
}

Consider the following example graph comprising vertices $A, B, C, D, E$ and edges $a, b, c, d$.
\begin{center}
	\tikzstyle{spot node}=[
			circle,
			very thick,
			draw=blue,
			text=blue,
			edge from parent/.style={red,draw},
		]
	\tikzstyle{edge link}=[
			text=red
		]
	
	\begin{tikzpicture}[
			level distance=14mm,
	    edge from parent/.style={red,very thick,draw},
			->]
	  \node [spot node] {$A$}
	  child {
			node [spot node] {$B$}
		  child [level distance=10mm, sibling distance=25mm ] {
				node [spot node] {$C$}
				edge from parent node [above left] {$b$}
			}
		  child [level distance=10mm, sibling distance=25mm ] {
				node [spot node] {$D$}
			  child [level distance=14mm] {
					node [spot node] {$E$}
					edge from parent node [edge link, right] {$d$}
				}
				edge from parent node [above right] {$c$}
			}
			edge from parent node [right] {$a$}
		};
	\end{tikzpicture}
\end{center}
This is laid out in memory as follows
\begin{center}
	\begin{tikzpicture}
		\coordinate (s);
		\spot {$-1$}  {$0$}   {0}{A}
		\spot  {$0$}  {$1$}   {1}{B}
		\spot  {$1$} {$-1$}   {2}{C}
		\spot  {$2$}  {$3$}   {3}{D}
		\spot  {$3$} {$-1$}   {4}{E}
		
		\coordinate (e) at (5,0);
		\edge  {$0$}  {$1$} {$-1$} {$-1$}   {0}{a}
		\edge  {$1$}  {$2$}  {$2$} {$-1$}   {1}{b}
		\edge  {$1$}  {$3$} {$-1$} {$-1$}   {2}{c}
		\edge  {$3$}  {$4$} {$-1$} {$-1$}   {3}{d}
		
		\tikzstyle{edge loop}=[
			to path={.. controls +(180:1) and +(-200:1) .. (\tikztotarget) \tikztonodes}
		]
	    
	  \draw [incoming edge] (sBc1.east) to [out=0, in=180] ($(eac1.west)-(2mm,0mm)$);
	  \node [incoming edge] at (eac4.west) {$\bot$};

	  \draw [outgoing edge] (sBc2.east) to [out=0, in=180] ($(ebc1.west)-(2mm,0mm)$);
	  \draw [outgoing edge] (ebc3.west) to [edge loop] ($(ecc1.west)-(2mm,0mm)$);
	  \node [outgoing edge] at (ecc3.west) {$\bot$};

		\begin{pgfonlayer}{background}
	  	\node [incoming bg, fit=(sBc1)] {};
	  	\node [incoming bg, fit=(eac4)] {};
	  	\node [outgoing bg, fit=(sBc2)] {};
	  	\node [outgoing bg, fit=(ebc3)] {};
	  	\node [outgoing bg, fit=(ecc3)] {};
	  \end{pgfonlayer}
	\end{tikzpicture}
\end{center}
The links between vertex $B$ and it's adjacent edges $a, b, c$ have been highlighted.
Let's look at that in more detail:
Vertex $B$ is stored at element index $1$ in the vertex memory array.
$B$ has one incoming edge $a$.
The edge $a$ is stored at element index $0$ in the edge memory array.
Therefore the \texttt{FIRST\_IN\_EDGE} field of $B$ is $0$.
%
Apart from $a$, the vertex $B$ has no further incoming edges.
Therefore, the \texttt{NEXT\_TARGET\_EDGE} field of $a$ is $-1$, i.e., the list of edges entering $B$ terminates here.

$B$ has two outgoing edges $b, c$.
The edge $b$ is stored at element index $1$ in the edge memory array.
Therefore the \texttt{FIRST\_OUT\_EDGE} field of $B$ is $1$.
%
The next outgoing edge of $B$ is $c$ which is stored at element index $2$.
Therefore, the \texttt{NEXT\_SOURCE\_EDGE} field of $b$ is $2$.
After $c$, the vertex $B$ has no further outgoing edges.
Therefore, the \texttt{NEXT\_SOURCE\_EDGE} field of $c$ is $-1$, i.e., the list of edges leaving $B$ terminates here.

Below is the same memory layout again, this time highlighting the references from edges $a, b, c$ back to the vertex memory array.
\begin{center}
	\begin{tikzpicture}
		\coordinate (s);
		\spot {$-1$}  {$0$}   {0}{A}
		\spot  {$0$}  {$1$}   {1}{B}
		\spot  {$1$} {$-1$}   {2}{C}
		\spot  {$2$}  {$3$}   {3}{D}
		\spot  {$3$} {$-1$}   {4}{E}
		
		\coordinate (e) at (5,0);
		\edge  {$0$}  {$1$} {$-1$} {$-1$}   {0}{a}
		\edge  {$1$}  {$2$}  {$2$} {$-1$}   {1}{b}
		\edge  {$1$}  {$3$} {$-1$} {$-1$}   {2}{c}
		\edge  {$3$}  {$4$} {$-1$} {$-1$}   {3}{d}
		
	  \draw [outgoing edge] ($(eac1.west)-(2mm,0mm)$) to [out=180, in=0]   (sAc1.east);
	  \draw [incoming edge]   (eac2.west)             to [out=180, in=0] ($(sBc1.east)+(0mm,1mm)$);

	  \draw [outgoing edge] ($(ebc1.west)-(2mm,0mm)$) to [out=180, in=0]   (sBc1.east);
	  \draw [incoming edge]   (ebc2.west)             to [out=180, in=0]   (sCc1.east);

	  \draw [outgoing edge] ($(ecc1.west)-(2mm,0mm)$) to [out=180, in=0] ($(sBc1.east)-(0mm,1mm)$);
	  \draw [incoming edge]   (ecc2.west)             to [out=180, in=0]   (sDc1.east);
	\end{tikzpicture}
\end{center}
For example, edge $c$ is leaving vertex $B$ (index $1$) and entering vertex $D$ (index $3$).
Therefore the \texttt{SOURCE} field of $c$ is $1$, and the \texttt{TARGET} field of $c$ is $3$.





% -----------------------------------------
\subsection{Free-list of unallocated elements.}
% -----------------------------------------

\tikzstyle{free bg}=[
		fill=gray!20,
		inner sep=0pt
	]

\newcommand{\freespot}[2]{
  \matrix (s) at (s.south west) [spot matrix]
  {
    \node (s#2c1) {$\mathbf{-2}$}; \\
    \node (s#2c2) {#1}; \\
    \node (s#2c3) [minimum height=13.5mm] {}; \\
  };
	\node at (s#2c1.west) [text width=3mm, draw=none, xshift=-1mm, text=blue, font=\scriptsize] {#2};
	\begin{pgfonlayer}{background}
  	\node [free bg, fit=(s#2c1)(s#2c3)] {};
  \end{pgfonlayer}
}

\newcommand{\freeedge}[2]{
  \matrix (e) at (e.south west) [spot matrix]
  {
  	\node (e#2c1) {$\mathbf{-2}$}; \\
    \node (e#2c2) {#1}; \\
    \node (e#2c3) [minimum height=19mm] {}; \\
  };
	\node at (e#2c1.west) [text width=3mm, draw=none, xshift=-1mm, text=red, font=\scriptsize] {#2};
	\begin{pgfonlayer}{background}
  	\node [free bg, fit=(e#2c1)(e#2c3)] {};
  \end{pgfonlayer}
}


The vertex and edge memory arrays can only ever grow.
When elements are released, they are simply marked as free for re-use.
Assume that in the above example vertex $D$ is deleted, as well as it's adjacent edges $c, d$, leaving this:
\begin{center}
	\tikzstyle{spot node}=[
			circle,
			very thick,
			draw=blue,
			text=blue,
			edge from parent/.style={red,draw},
		]
	\tikzstyle{edge link}=[
			text=red
		]
	
	\begin{tikzpicture}[
			level distance=14mm,
	    edge from parent/.style={red,very thick,draw},
			->]
	  \node [spot node] {$A$}
	  child {
			node [spot node] {$B$}
		  child [level distance=10mm, sibling distance=25mm ] {
				node [spot node] {$C$}
				edge from parent node [above left] {$b$}
			}
		  child [level distance=10mm, sibling distance=25mm] {
				node [spot node, draw=gray!40, text=gray!40] {$D$}
			  child [level distance=14mm] {
					node [spot node] {$E$}
					edge from parent [draw=gray!40, text=gray!40] node [right] {$d$}
				}
				edge from parent [draw=gray!40, text=gray!40] node [above right] {$c$}
			}
			edge from parent node [right] {$a$}
		};
	\end{tikzpicture}
\end{center}

\noindent
After removing $c, d, D$ the memory layout looks like this:

\begin{center}
	\begin{tikzpicture}
		\coordinate (s);
		\spot {$-1$}  {$0$}   {0}{A}
		\spot  {$0$}  {$1$}   {1}{B}
		\spot  {$1$} {$-1$}   {2}{C}
		\freespot {$-1$} {3}
		\spot {$-1$} {$-1$}   {4}{E}
		
		\coordinate (e) at (5,0);
		\edge  {$0$}  {$1$} {$-1$} {$-1$}   {0}{a}
		\edge  {$1$}  {$2$} {$-1$} {$-1$}   {1}{b}
		\freeedge {$-1$} {2}
		\freeedge {$2$} {3}
		
		\tikzstyle{edge loop}=[
			to path={.. controls +(180:1.5) and +(-170:1.5) .. (\tikztotarget) \tikztonodes}
		]
		\tikzstyle{freed edge}=[
				black,
				thick,
				->,
				shorten >=1mm,
				shorten <=1mm,
				anchor=east
			]
	    
		\node [freed edge] (ffs) at ($(s3c1.west)-(10mm,0mm)$) {\texttt{free}};
	  \draw [freed edge, shorten <=0mm] (ffs) to ($(s3c1.west)-(2mm,0mm)$);
	  \node [freed edge] at (s3c2.west) {$\bot$};

		\node [freed edge] (ffe) at ($(e3c1.west)-(13mm,0mm)$) {\texttt{free}};
	  \draw [freed edge, shorten <=0mm] (ffe) to ($(e3c1.west)-(2mm,0mm)$);
	  \draw [freed edge] (e3c2.west) to [edge loop] ($(e2c1.west)-(2mm,0mm)$);
	  \node [freed edge] at (e2c2.west) {$\bot$};
	  
		\node [draw=none] at ($(e2c1.east)+(18mm,0mm)$) {};	  
	\end{tikzpicture}
\end{center}
The element $3$ in the vertex memory array as well as elements $2$ and $3$ in the edge memory array have been marked as free.
This is done by putting the magic number ``$-2$'' into the first 4 bytes of the element.\footnote{
	In vertices and edges the first 4 bytes are always occupied by a (positive) index or a $-1$ index list terminator.
	Therefore, occupied and free blocks can not be confused.}
The next 4 bytes of a freed element are the index of the next freed element in the (same) memory array,
	or $-1$ if there is no next freed element.
Each memory array remembers the index \texttt{free} of the first freed element.

Newly freed elements are enqueued at \texttt{free}, that is, at the head of the free-list.
So in the above example, edge element $2$ was freed first, followed by edge element $3$.

The next edge element will be allocated at head of the free-list and the \texttt{free} index move to the next element.
For example, assume that a new edge is created from $B$ to $E$:
\begin{center}
	\begin{tikzpicture}
		\coordinate (s);
		\spot {$-1$}  {$0$}   {0}{A}
		\spot  {$0$}  {$1$}   {1}{B}
		\spot  {$1$} {$-1$}   {2}{C}
		\freespot {$-1$} {3}
		\spot {$3$} {$-1$}   {4}{E}
		
		\coordinate (e) at (5,0);
		\edge  {$0$}  {$1$} {$-1$} {$-1$}   {0}{a}
		\edge  {$1$}  {$2$} {$3$} {$-1$}   {1}{b}
		\freeedge {$-1$} {2}
		\edge  {$1$}  {$4$} {$-1$} {$-1$}   {3}{ }
		
		\tikzstyle{edge loop}=[
			to path={.. controls +(180:1.5) and +(-170:1.5) .. (\tikztotarget) \tikztonodes}
		]
		\tikzstyle{freed edge}=[
				black,
				thick,
				->,
				shorten >=1mm,
				shorten <=1mm,
				anchor=east
			]
	    
		\node [freed edge] (ffs) at ($(s3c1.west)-(10mm,0mm)$) {\texttt{free}};
	  \draw [freed edge, shorten <=0mm] (ffs) to ($(s3c1.west)-(2mm,0mm)$);
	  \node [freed edge] at (s3c2.west) {$\bot$};

		\node [freed edge] (ffe) at ($(e2c1.west)-(13mm,0mm)$) {\texttt{free}};
	  \draw [freed edge, shorten <=0mm] (ffe) to ($(e2c1.west)-(2mm,0mm)$);
	  \node [freed edge] at (e2c2.west) {$\bot$};
	  
		\node [draw=none] at ($(e2c1.east)+(18mm,0mm)$) {};	  
	\end{tikzpicture}
\end{center}
If \texttt{free} is -1, then no more freed elements are available and the underlying memory array has to grow to fit newly added elements.
