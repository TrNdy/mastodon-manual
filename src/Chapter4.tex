\section{Getting your bearings in large datasets.}

In the previous chapter we have seen how to edit single spots and links in Mastodon, what can be called point-wise editing.
As we said before, the goal of Mastodon is to let you harness very large images, for which the number of annotations can be very large too.
It can be very easy to get lost within such large images and loose track of where we are within the sample image and what cell we follow. 
So we have added several features that are made especially to get your bearings in large datasets.
More than anything, these features are about giving visual cues that ease orientation, and exploit event that a human eye would consider salient. 
We took some inspiration from video-games, that are very good at communicating condensed and synthetic information to the player
(but only for a limited part; there is no screenshake when you delete a spot).



\subsection{Bookmarks in the BDV views.}

\subsection{Linking several views together.}

\subsection{In Mastodon everything is animated.}

\subsection{Spatial context in TrackScheme.}